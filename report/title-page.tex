\begin{titlepage}
	Universität Passau\newline
	Fakultät für Informatik und Mathematik
	\vspace{2.5cm}
    \begin{center}
    \LARGE\textbf{{Classification  Of Visualization  In Scientific Literature}}\\
   
    \normalsize

    \vspace{2.5cm}
    \end{center}

 \normalsize{
 	Masterarbeit zur Erlangung des akademischen Grades\newline
 	Master of Science (M.Sc.)\newline
 	\ \\
 	Lehrstuhl für Intelligent Systems und Lehrstuhl für Data Science \newline
 	der Fakultät für Informatik und Mathematik\newline
 	der Universität Passau\newline
 	
 
    \begin{tabular}{ll}
    	Name: & Arnold Azeem \\
    	Matrikelnummer: & 79176 \\
    	Fachbereich: & Informatik\\
    	Studiengang: & Master Informatik\\
	Erstprüfer: & Prof. Dr. Christin Siefert \\
	Zweitprüfer: & Prof. Dr. Michael Granitzer\\
	Date: &     \today
    \end{tabular}\\
    }



\newpage
\begin{abstract}
Chart image classification serves as an initial step towards comprehending, extracting and analysing data described with charts. 
These charts are regularly embedded in research papers and journals and, are used to describe research findings. Data is visualized in different forms and patterns, and there exist no strict standards to follow. Chart classification consists of interpreting visual content and identifying the chart type. Previous approaches for dealing with chart classification depended on hand crafted features which is not robust since charts can be presented using variable structures.
Therefore, to achieve this goal, this thesis proposes using CovNets, a deep learning-based approach that automates the feature acquisition step. We use a CovNet architecture inspired by the AlexNet CovNet architecture.~\cite{krizhevsky2012imagenet} for training a model for classification, 4 classes of chart images (Scatter Plot, Bar Chart, Box plot and Line Chart) are annotated with 16,000 chart images. With an accuracy of ... our proposed method is very efficient.

\end{abstract}

\renewcommand{\abstractname}{Acknowledgements}

\end{titlepage}
\begin{titlepage}
	Universität Passau\newline
	Fakultät für Informatik und Mathematik
	\vspace{2.5cm}
    \begin{center}
    \LARGE\textbf{{Classification  Of Visualization  In Scientific Literature}}\\
   
    \normalsize

    \vspace{2.5cm}
    \end{center}

 \normalsize{
 	Masterarbeit zur Erlangung des akademischen Grades\newline
 	Master of Science (M.Sc.)\newline
 	\ \\
 	Lehrstuhl für Intelligent Systems und Lehrstuhl für Data Science \newline
 	der Fakultät für Informatik und Mathematik\newline
 	der Universität Passau\newline
 	
 
    \begin{tabular}{ll}
    	Name: & Arnold Azeem \\
    	Matrikelnummer: & 79176 \\
    	Fachbereich: & Informatik\\
    	Studiengang: & Master Informatik\\
	Erstprüfer: & Prof. Dr. Christin Siefert \\
	Zweitprüfer: & Prof. Dr. Michael Granitzer\\
	Date: &     \today
    \end{tabular}\\
    }

\newpage
\section{Abstract}
Distinct visualization techniques are used in scientific research publications to summarize large amount of data and also represent a variety of data. These visualizations help to communicate complex information and support the arguments being presented in the publication in a way that is easy to understand and follow.
These figures tend to reveal trends, patterns or relations that might have otherwise be difficult to grasp using only text. 
It is therefore relevant that we extract the data from these visualizations since the extracted data can be used for validating the publication or presenting the data in another form for a different audience. In this context, classifying these visualizations is the initial step since, there is a variety of visualizations and each one is processed in a specific way. It is only after classification that extracting of raw data from these visualizations can be acquired for other tasks. This thesis presents an approach whereby real world data is used to create four types of plots (scatter plots, bar charts, line charts, and box-plots) and random plots also of the same kind from the Internet are added together and used to train and evaluate a CovNet model to be able to classify these plots. 


 



\end{titlepage}
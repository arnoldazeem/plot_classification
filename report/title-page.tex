\begin{titlepage}
	Universität Passau\newline
	Fakultät für Informatik und Mathematik
	\vspace{2.5cm}
    \begin{center}
    \LARGE\textbf{{Classification  Of Visualization  In Scientific Literature}}\\
   
    \normalsize

    \vspace{2.5cm}
    \end{center}

 \normalsize{
 	Masterarbeit zur Erlangung des akademischen Grades\newline
 	Master of Science (M.Sc.)\newline
 	\ \\
 	Lehrstuhl für Intelligent Systems und Lehrstuhl für Data Science \newline
 	der Fakultät für Informatik und Mathematik\newline
 	der Universität Passau\newline
 	
 
    \begin{tabular}{ll}
    	Name: & Arnold Azeem \\
    	Matrikelnummer: & 79176 \\
    	Fachbereich: & Informatik\\
    	Studiengang: & Master Informatik\\
	Erstprüfer: & Prof. Dr. Christin Siefert \\
	Zweitprüfer: & Prof. Dr. Michael Granitzer\\
	Date: &     \today
    \end{tabular}\\
    }



\newpage
\begin{abstract}
Chart image classification serves as an initial step towards comprehending, extracting and further analysing data described with charts. These chart images are regularly embedded in scientific papers and journals to describe research findings. 
An immediate problem is that, charts images have different forms and patterns, and there exist ancillary structures like text, legends and axis, this makes the task challenging.
Chart classification consists of extracting chart features which are then used for identifying the chart type.
Previous approaches for dealing with chart classification depended on hand crafted features which is not robust when dealing with large amount of data with variable content structure.
This thesis therefore, proposes using CovNets, a deep learning-based approach that automates the feature acquisition step. We use a CovNet architecture inspired by the AlexNet CovNet architecture.~\cite{krizhevsky2012imagenet} for training a model for classification, a 4 class chart image dataset consisting of 17,357 chart images. With an accuracy of 84\% our proposed method is very efficient.

\end{abstract}

\section*{Acknowledgement}
I would like to thank everyone who made this master thesis possible. First of all, I greatly appreciate the help of my supervisor Prof. Christin Siefert now of University of Twente who advised me all the way during this thesis work. I specially thank her for her infinite patience. The discussions I had with her were invaluable.
I want to say a big thanks to my second Supervisor Prof. Michael Granitzer. Besides, I am grateful to all the Passau teaching staff and students who generously
contributed one way or another.
My final words go to my family. I want to thank my family, for their love and support all
these years.\\
Arnold Azeem, Passau 13/08/2018

\end{titlepage}
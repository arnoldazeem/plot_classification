\documentclass[12pt, a4paper,oneside]{report}



\usepackage{titlesec}
\usepackage[utf8]{inputenc}
\usepackage{german}
\usepackage[T1]{fontenc}
\usepackage{textcomp}
\usepackage{framed}
\usepackage{pbox}
\usepackage{caption}
\usepackage[nottoc,numbib]{tocbibind}
\usepackage[pdftex]{graphicx}
\usepackage[toc]{glossaries}
\usepackage{amssymb, amsmath}
\usepackage[colorlinks,
pdfstartview = FitH,
linkcolor = black,
plainpages = false,
hypertexnames = false,
citecolor = black]{hyperref}
\usepackage{setspace}
\onehalfspacing
\usepackage[left=3.5cm,right=3.5cm,top=2cm,bottom=2cm]{geometry}
\graphicspath{{./pics/}}
\usepackage[printonlyused]{acronym}
\usepackage{todonotes}
\usepackage{subcaption}
\usepackage{float}
\usepackage{pifont}
\newcommand{\cmark}{\ding{51}}%
\newcommand{\xmark}{\ding{55}}%
\usepackage{multirow}
\usepackage{pdflscape}
\usepackage{eurosym}

\usepackage{tocbibind}



\parindent 0pt

\begin{document}

\begin{titlepage}
	Universität Passau\newline
	Fakultät für Informatik und Mathematik
	\vspace{2.5cm}
    \begin{center}
    \LARGE\textbf{{Classification  Of Visualization  In Scientific Literature}}\\
   
    \normalsize

    \vspace{2.5cm}
    \end{center}

 \normalsize{
 	Masterarbeit zur Erlangung des akademischen Grades\newline
 	Master of Science (M.Sc.)\newline
 	\ \\
 	Lehrstuhl für Intelligent Systems und Lehrstuhl für Data Science \newline
 	der Fakultät für Informatik und Mathematik\newline
 	der Universität Passau\newline
 	
 
    \begin{tabular}{ll}
    	Name: & Arnold Azeem \\
    	Matrikelnummer: & 79176 \\
    	Fachbereich: & Informatik\\
    	Studiengang: & Master Informatik\\
	Erstprüfer: & Prof. Dr. Christin Siefert \\
	Zweitprüfer: & Prof. Dr. Michael Granitzer\\
	Date: &     \today
    \end{tabular}\\
    }



\newpage
\begin{abstract}
Distinct visualisation techniques are used in scientific research publications to summarise large amount of data and also represent a variety of data. These visualisations help to communicate complex information and support the arguments being presented in the publication in a way that is easy to understand and follow.
These figures tend to reveal trends, patterns or relations that might have otherwise been difficult to grasp using only text. 
It is therefore relevant that we extract the data from these visualisations since the extracted data can be used for validating the publication or presenting the data in another form for a different audience. In this context, classifying these visualisations is the initial step since, there is a variety of visualisations and each one is processed in a specific way. It is only after classification that extraction of raw data from these visualisations can be acquired for other tasks. This thesis presents an approach whereby real world data is used to create four types of plots (scatter plots, bar charts, line charts, and box-plots) and random plots also of the same kind from the Internet are added together and used to train and evaluate a CovNet model to be able to classify these plots. 
\end{abstract}

\renewcommand{\abstractname}{Acknowledgements}

\end{titlepage}

\setcounter{tocdepth}{10}
\tableofcontents


% \begin{acronym}

% \acro{XML} {Extensible Markup Language}s

% \end{acronym}


\listoffigures
\listoftables

\titleformat{\chapter}{\LARGE\bfseries}{\thechapter}{1em}{}



\chapter*{Abstract}


\newpage
\chapter{Einleitung}

\section{Motivation}

\begin{figure}[!ht]
	\centering
% remove these comments to insert picture at img/file.png
% \includegraphics[width=1.0\textwidth]{img/file.png}
	\caption{Describe this picture.}
	\label{fig:1}
\end{figure}

\chapter{ABSTRACT}
Distinct visualization techniques are used in scientific research publications to summarize large amount of data and also represent a variety of data. These visualizations help to communicate complex information and support the arguments presented in the paper in a easy to understand and follow way.
These figures tend to reveal trends,patterns or relations that might otherwise be difficult to grasp using only text. In this context, classifying these visualizations is really relevant since there is a variety of visualizations and each one will have a different approach to processing it, example is extracting the raw data from it. 

\chapter{INTRODUCTION}
A picture tells a thousand words even though a cliché stands to be very true especially when it comes to presenting complex findings in scientific research publications. The importance of these figures in papers cannot be undermined since they provide a way to easily interpret,find patterns and relations in the data which would have otherwise been more complex relying on  only textual data. Processing of these figures could lead to data extraction and analysis which would further improve research. In this work we present a way to automatically detect various types of graphs. This goes a long way to assist in further processing since each type of visualization allows a specific way of processing. 


\chapter{MOTIVATION}
Complex data is better explained in scientific papers with the aid of visualizations. These plots present complex data in an easy to understand way compared to textual representation. The data which these visualizations contain when extracted play an important role in events where another researcher wants to verify the work of the publisher, this data can also be used to develop other visualizations in situations where the paper needs to be presented to a different audience with a different background as opposed to the audience which the visualizations were created for, Also when comparing two plots the raw data helps make a better decision than just the figures. 


The diagram below shows vision of this work, starts from getting the research paper to the point
where the raw data is obtained. But this work mainly focuses on the red dotted lines shown below in
the diagram.


\chapter{LITERATURE REVIEW}



\chapter{CREATING PLOTS}
The dataset consists of images of plots which are automatically created and labeled. Scripts are created in Python, Matlab,R language and Java. This is done to make our model very robust since plots used in scientific papers could be created in any of the above programming languages. The dummy data used in creating these plots are all in csv format. All the csv files are put into one folder and each is read at a time. For each csv file the script reads the first column and discards it if it contains any string values except in its header. Its saves this column into an array as the x-axis and saves the next column into an array as the y-axis if it does not also contain string values in the column. This two arrays are then used to 
create the plots, their headers serve as the label of their axis and the name of the file serve as the title. The previous
y-axis becomes the new x-axis and the next column becomes the y-axis and the process continues till the columns in the csv file are all used. So depending on the number of numeric columns in the csv file many plots are created and labeled.


\chapter{Scatter plot}
In other to capture all variants of scatter plots in research papers I looked through the dataset of this [] paper and this paper []. From the information gotten from those papers i created scatter plots which are positive correlated, negatively correlated and scatter plots with random plots. I also created plots with different markers as the points, some of the markers are circles,triangles,squares,stars etc., since not all markers all not used in scientific papers the above mentioned markers have a higher occurrence than other markers. Another realization was that some of the markers were filled and others were not, also other scatter plots had legends in them to describe the dataset a little more, all these variants were captured in the plots created. This is done to capture the different ways in which scatter plots are represented in scientific research papers.


\chapter{Bar Chart}





\chapter{Scatter plot}
For constructing the plots for the scatter plots I looked through a dataset from this paper[]. The dataset contained 555
scatter plots and looking through them I realized the markers used to plot where mainly zeros, triangles,squares,dots and x's.
So instead of just plotting all the markers randomly i plotted those mentioned above with a higher probability than the rest of the markers.
Also the scatter plots consisted of positive and negative correlation and just a few zero correlation. 
I also noticed that some markers were filled and other were not filled. So for creating my dataset I plotted with some markers being filled and other unfilled and this was done randomly. 

\subsection{Dataset for Scatter plot}
For plotting my scatter plots I used raw data correlated into csv files. Next I will describe the data used and their sources.
This group of data was gotten from [vincentarel].

The first is the \textbf{loti.csv} file which contains temperature anomalies for the years 1880 to 2010 and this data is from GISS(Goddard Institute for Space Studies) Land-Ocean
Temperature Index(LOTI) data. It has data for months of the year and averages of some combined months.
Second raw data used is \textbf{USjudgerAtings.csv} contains the ratings of 43 state judges in the US Superior Court by some lawyers. The rating were given based on attributes like Diligence,integrity,demeanor etc.
Third data used is the famous Iris dataset by Edgar Anderson
This dataset gives the measurements in centimeters of the variables sepal length and width and petal length and width for 50 flowers of 3 different species of iris; setosa,versicolor and virginica. 
The next data Longley's Economic Regression Data which was good because it gave very good positive correlation scatter plots even though it was not as huge as the other dataset used. It
consisted of 7 economical variables observed yearly from 1947 to 1962.
The next dataset is called the quakes, which has information of locations of Earthquakes off Fiji. It has 1000 observations with 5 variables(longitude,latitude, depth,Richter Magnitude and number of stations reporting).
The nuclear Power station Construction Data was also used. It contains information about the construction of 32 light water reactor plants in the U.S.A  in the 90's.

The second group of data was taken from [Datplot website] a website for a software called DatPlot for plotting raw data.












\begin{thebibliography}{12345678}
\bibitem[BDE-1, o.J.]{BDE-1} "'The Big List of D3.js Examples"', http://christopheviau.com/d3list/, 20.01.2015






\end{thebibliography}

\newpage
\chapter*{Erklärung zur Masterarbeit}
Ich erkläre hiermit, dass ich die vorliegende Arbeit selbständig verfasst und keine anderen als die angegebenen Quellen und Hilfsmittel verwendet habe. \newline
\ \\
Die Arbeit wurde bisher keiner anderen Prüfungsbehörde vorgelegt und auch noch nicht veröffentlicht.\newline
\ \\
Passau, den <date>
\newline
\ \\
\ \\
<First Name, Last Name>
\end{document}